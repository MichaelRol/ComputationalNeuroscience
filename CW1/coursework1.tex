\documentclass[12pt]{article}
\usepackage{amsfonts, epsfig}

\usepackage{graphicx}
\usepackage{fancyhdr}
\pagestyle{fancy}
\lfoot{\texttt{https://github.com/coms30127}}
\lhead{Computation Neuroscience - Coursework 1 - auto-associative networks}
\rhead{\thepage}
\cfoot{}
\begin{document}

\section*{Coursework 1}

This coursework relates to the Hopfield network. In the \texttt{coursework1}
folder you will find programmes \texttt{seven\_segment.jl} and
\texttt{seven\_segment.py}. These contain a function for converting an
11-component vector of ones and -1's into an old-fashioned
seven-segment digit\footnote{\texttt{https://en.wikipedia.org/wiki/Seven-segment\_display}} and a number: the first seven digits correspond to
the seven-segment display and the remaining four code for the number
in a sort of binary where the zeros have been replaced with -1. It
also contains three patterns: one, three and six.

The goal of this coursework is to extend one or other of these
programmes to include a Hopfield network to store these three
patterns using the formula for $w_{ij}$ when $i\not=j$
\begin{equation}
w_{ij}=\frac{1}{N}\sum_a x_i^a x_j^a
\end{equation}
where $a$ labels patterns, $x_i^a$ is the activiation of the $i$th
node in the $a$th pattern and $N$ is the number of
patterns. $w_{ii}=0$.

In the code there are also two test patterns, your programme should
update these synchronously until convergence and print out the
patterns at each iteration. 


This coursework is intended to check you understand Hopfield networks,
you are not being asked to do anything over elaborate beyond that. In
summary you should:
\begin{itemize}
\item Create a weight matrix of $w_{ij}$s.
\item Fix the weight values to store the three patterns.
\item Write a function to evolve the network according to the
  McCulloch-Pitts formula; this should be done synchronously so all the nodes are updated at each timestep.
\item For each of the two test patterns, evolve the patterns until they stop changing, printing the pattern at each step.
\end{itemize}




\subsection*{COMSM20127}

Add to the above a function to calculate the energy of a configuration
\begin{equation}
E=-\frac{1}{2}\sum_{ij} x_i w_{ij} x_j
\end{equation}
and print out the energy of the three learned patterns, of the test
patterns and any patterns formed as the patterns are updated.

\subsection*{Submission instructions}

[We are finalising the submission instructions and will update this pdf shortly. The submission will likely be in the form of a brief 2 or 3 page report via SAFE.]

The submission deadline is 5pm on Monday 11th March 2019.

\end{document}
